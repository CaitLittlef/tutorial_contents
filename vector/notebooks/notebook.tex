
% Default to the notebook output style

    


% Inherit from the specified cell style.




    
\documentclass[11pt]{article}

    
    
    \usepackage[T1]{fontenc}
    % Nicer default font (+ math font) than Computer Modern for most use cases
    \usepackage{mathpazo}

    % Basic figure setup, for now with no caption control since it's done
    % automatically by Pandoc (which extracts ![](path) syntax from Markdown).
    \usepackage{graphicx}
    % We will generate all images so they have a width \maxwidth. This means
    % that they will get their normal width if they fit onto the page, but
    % are scaled down if they would overflow the margins.
    \makeatletter
    \def\maxwidth{\ifdim\Gin@nat@width>\linewidth\linewidth
    \else\Gin@nat@width\fi}
    \makeatother
    \let\Oldincludegraphics\includegraphics
    % Set max figure width to be 80% of text width, for now hardcoded.
    \renewcommand{\includegraphics}[1]{\Oldincludegraphics[width=.8\maxwidth]{#1}}
    % Ensure that by default, figures have no caption (until we provide a
    % proper Figure object with a Caption API and a way to capture that
    % in the conversion process - todo).
    \usepackage{caption}
    \DeclareCaptionLabelFormat{nolabel}{}
    \captionsetup{labelformat=nolabel}

    \usepackage{adjustbox} % Used to constrain images to a maximum size 
    \usepackage{xcolor} % Allow colors to be defined
    \usepackage{enumerate} % Needed for markdown enumerations to work
    \usepackage{geometry} % Used to adjust the document margins
    \usepackage{amsmath} % Equations
    \usepackage{amssymb} % Equations
    \usepackage{textcomp} % defines textquotesingle
    % Hack from http://tex.stackexchange.com/a/47451/13684:
    \AtBeginDocument{%
        \def\PYZsq{\textquotesingle}% Upright quotes in Pygmentized code
    }
    \usepackage{upquote} % Upright quotes for verbatim code
    \usepackage{eurosym} % defines \euro
    \usepackage[mathletters]{ucs} % Extended unicode (utf-8) support
    \usepackage[utf8x]{inputenc} % Allow utf-8 characters in the tex document
    \usepackage{fancyvrb} % verbatim replacement that allows latex
    \usepackage{grffile} % extends the file name processing of package graphics 
                         % to support a larger range 
    % The hyperref package gives us a pdf with properly built
    % internal navigation ('pdf bookmarks' for the table of contents,
    % internal cross-reference links, web links for URLs, etc.)
    \usepackage{hyperref}
    \usepackage{longtable} % longtable support required by pandoc >1.10
    \usepackage{booktabs}  % table support for pandoc > 1.12.2
    \usepackage[inline]{enumitem} % IRkernel/repr support (it uses the enumerate* environment)
    \usepackage[normalem]{ulem} % ulem is needed to support strikethroughs (\sout)
                                % normalem makes italics be italics, not underlines
    

    
    
    % Colors for the hyperref package
    \definecolor{urlcolor}{rgb}{0,.145,.698}
    \definecolor{linkcolor}{rgb}{.71,0.21,0.01}
    \definecolor{citecolor}{rgb}{.12,.54,.11}

    % ANSI colors
    \definecolor{ansi-black}{HTML}{3E424D}
    \definecolor{ansi-black-intense}{HTML}{282C36}
    \definecolor{ansi-red}{HTML}{E75C58}
    \definecolor{ansi-red-intense}{HTML}{B22B31}
    \definecolor{ansi-green}{HTML}{00A250}
    \definecolor{ansi-green-intense}{HTML}{007427}
    \definecolor{ansi-yellow}{HTML}{DDB62B}
    \definecolor{ansi-yellow-intense}{HTML}{B27D12}
    \definecolor{ansi-blue}{HTML}{208FFB}
    \definecolor{ansi-blue-intense}{HTML}{0065CA}
    \definecolor{ansi-magenta}{HTML}{D160C4}
    \definecolor{ansi-magenta-intense}{HTML}{A03196}
    \definecolor{ansi-cyan}{HTML}{60C6C8}
    \definecolor{ansi-cyan-intense}{HTML}{258F8F}
    \definecolor{ansi-white}{HTML}{C5C1B4}
    \definecolor{ansi-white-intense}{HTML}{A1A6B2}

    % commands and environments needed by pandoc snippets
    % extracted from the output of `pandoc -s`
    \providecommand{\tightlist}{%
      \setlength{\itemsep}{0pt}\setlength{\parskip}{0pt}}
    \DefineVerbatimEnvironment{Highlighting}{Verbatim}{commandchars=\\\{\}}
    % Add ',fontsize=\small' for more characters per line
    \newenvironment{Shaded}{}{}
    \newcommand{\KeywordTok}[1]{\textcolor[rgb]{0.00,0.44,0.13}{\textbf{{#1}}}}
    \newcommand{\DataTypeTok}[1]{\textcolor[rgb]{0.56,0.13,0.00}{{#1}}}
    \newcommand{\DecValTok}[1]{\textcolor[rgb]{0.25,0.63,0.44}{{#1}}}
    \newcommand{\BaseNTok}[1]{\textcolor[rgb]{0.25,0.63,0.44}{{#1}}}
    \newcommand{\FloatTok}[1]{\textcolor[rgb]{0.25,0.63,0.44}{{#1}}}
    \newcommand{\CharTok}[1]{\textcolor[rgb]{0.25,0.44,0.63}{{#1}}}
    \newcommand{\StringTok}[1]{\textcolor[rgb]{0.25,0.44,0.63}{{#1}}}
    \newcommand{\CommentTok}[1]{\textcolor[rgb]{0.38,0.63,0.69}{\textit{{#1}}}}
    \newcommand{\OtherTok}[1]{\textcolor[rgb]{0.00,0.44,0.13}{{#1}}}
    \newcommand{\AlertTok}[1]{\textcolor[rgb]{1.00,0.00,0.00}{\textbf{{#1}}}}
    \newcommand{\FunctionTok}[1]{\textcolor[rgb]{0.02,0.16,0.49}{{#1}}}
    \newcommand{\RegionMarkerTok}[1]{{#1}}
    \newcommand{\ErrorTok}[1]{\textcolor[rgb]{1.00,0.00,0.00}{\textbf{{#1}}}}
    \newcommand{\NormalTok}[1]{{#1}}
    
    % Additional commands for more recent versions of Pandoc
    \newcommand{\ConstantTok}[1]{\textcolor[rgb]{0.53,0.00,0.00}{{#1}}}
    \newcommand{\SpecialCharTok}[1]{\textcolor[rgb]{0.25,0.44,0.63}{{#1}}}
    \newcommand{\VerbatimStringTok}[1]{\textcolor[rgb]{0.25,0.44,0.63}{{#1}}}
    \newcommand{\SpecialStringTok}[1]{\textcolor[rgb]{0.73,0.40,0.53}{{#1}}}
    \newcommand{\ImportTok}[1]{{#1}}
    \newcommand{\DocumentationTok}[1]{\textcolor[rgb]{0.73,0.13,0.13}{\textit{{#1}}}}
    \newcommand{\AnnotationTok}[1]{\textcolor[rgb]{0.38,0.63,0.69}{\textbf{\textit{{#1}}}}}
    \newcommand{\CommentVarTok}[1]{\textcolor[rgb]{0.38,0.63,0.69}{\textbf{\textit{{#1}}}}}
    \newcommand{\VariableTok}[1]{\textcolor[rgb]{0.10,0.09,0.49}{{#1}}}
    \newcommand{\ControlFlowTok}[1]{\textcolor[rgb]{0.00,0.44,0.13}{\textbf{{#1}}}}
    \newcommand{\OperatorTok}[1]{\textcolor[rgb]{0.40,0.40,0.40}{{#1}}}
    \newcommand{\BuiltInTok}[1]{{#1}}
    \newcommand{\ExtensionTok}[1]{{#1}}
    \newcommand{\PreprocessorTok}[1]{\textcolor[rgb]{0.74,0.48,0.00}{{#1}}}
    \newcommand{\AttributeTok}[1]{\textcolor[rgb]{0.49,0.56,0.16}{{#1}}}
    \newcommand{\InformationTok}[1]{\textcolor[rgb]{0.38,0.63,0.69}{\textbf{\textit{{#1}}}}}
    \newcommand{\WarningTok}[1]{\textcolor[rgb]{0.38,0.63,0.69}{\textbf{\textit{{#1}}}}}
    
    
    % Define a nice break command that doesn't care if a line doesn't already
    % exist.
    \def\br{\hspace*{\fill} \\* }
    % Math Jax compatability definitions
    \def\gt{>}
    \def\lt{<}
    % Document parameters
    \title{geopandas\_advanced}
    
    
    

    % Pygments definitions
    
\makeatletter
\def\PY@reset{\let\PY@it=\relax \let\PY@bf=\relax%
    \let\PY@ul=\relax \let\PY@tc=\relax%
    \let\PY@bc=\relax \let\PY@ff=\relax}
\def\PY@tok#1{\csname PY@tok@#1\endcsname}
\def\PY@toks#1+{\ifx\relax#1\empty\else%
    \PY@tok{#1}\expandafter\PY@toks\fi}
\def\PY@do#1{\PY@bc{\PY@tc{\PY@ul{%
    \PY@it{\PY@bf{\PY@ff{#1}}}}}}}
\def\PY#1#2{\PY@reset\PY@toks#1+\relax+\PY@do{#2}}

\expandafter\def\csname PY@tok@w\endcsname{\def\PY@tc##1{\textcolor[rgb]{0.73,0.73,0.73}{##1}}}
\expandafter\def\csname PY@tok@c\endcsname{\let\PY@it=\textit\def\PY@tc##1{\textcolor[rgb]{0.25,0.50,0.50}{##1}}}
\expandafter\def\csname PY@tok@cp\endcsname{\def\PY@tc##1{\textcolor[rgb]{0.74,0.48,0.00}{##1}}}
\expandafter\def\csname PY@tok@k\endcsname{\let\PY@bf=\textbf\def\PY@tc##1{\textcolor[rgb]{0.00,0.50,0.00}{##1}}}
\expandafter\def\csname PY@tok@kp\endcsname{\def\PY@tc##1{\textcolor[rgb]{0.00,0.50,0.00}{##1}}}
\expandafter\def\csname PY@tok@kt\endcsname{\def\PY@tc##1{\textcolor[rgb]{0.69,0.00,0.25}{##1}}}
\expandafter\def\csname PY@tok@o\endcsname{\def\PY@tc##1{\textcolor[rgb]{0.40,0.40,0.40}{##1}}}
\expandafter\def\csname PY@tok@ow\endcsname{\let\PY@bf=\textbf\def\PY@tc##1{\textcolor[rgb]{0.67,0.13,1.00}{##1}}}
\expandafter\def\csname PY@tok@nb\endcsname{\def\PY@tc##1{\textcolor[rgb]{0.00,0.50,0.00}{##1}}}
\expandafter\def\csname PY@tok@nf\endcsname{\def\PY@tc##1{\textcolor[rgb]{0.00,0.00,1.00}{##1}}}
\expandafter\def\csname PY@tok@nc\endcsname{\let\PY@bf=\textbf\def\PY@tc##1{\textcolor[rgb]{0.00,0.00,1.00}{##1}}}
\expandafter\def\csname PY@tok@nn\endcsname{\let\PY@bf=\textbf\def\PY@tc##1{\textcolor[rgb]{0.00,0.00,1.00}{##1}}}
\expandafter\def\csname PY@tok@ne\endcsname{\let\PY@bf=\textbf\def\PY@tc##1{\textcolor[rgb]{0.82,0.25,0.23}{##1}}}
\expandafter\def\csname PY@tok@nv\endcsname{\def\PY@tc##1{\textcolor[rgb]{0.10,0.09,0.49}{##1}}}
\expandafter\def\csname PY@tok@no\endcsname{\def\PY@tc##1{\textcolor[rgb]{0.53,0.00,0.00}{##1}}}
\expandafter\def\csname PY@tok@nl\endcsname{\def\PY@tc##1{\textcolor[rgb]{0.63,0.63,0.00}{##1}}}
\expandafter\def\csname PY@tok@ni\endcsname{\let\PY@bf=\textbf\def\PY@tc##1{\textcolor[rgb]{0.60,0.60,0.60}{##1}}}
\expandafter\def\csname PY@tok@na\endcsname{\def\PY@tc##1{\textcolor[rgb]{0.49,0.56,0.16}{##1}}}
\expandafter\def\csname PY@tok@nt\endcsname{\let\PY@bf=\textbf\def\PY@tc##1{\textcolor[rgb]{0.00,0.50,0.00}{##1}}}
\expandafter\def\csname PY@tok@nd\endcsname{\def\PY@tc##1{\textcolor[rgb]{0.67,0.13,1.00}{##1}}}
\expandafter\def\csname PY@tok@s\endcsname{\def\PY@tc##1{\textcolor[rgb]{0.73,0.13,0.13}{##1}}}
\expandafter\def\csname PY@tok@sd\endcsname{\let\PY@it=\textit\def\PY@tc##1{\textcolor[rgb]{0.73,0.13,0.13}{##1}}}
\expandafter\def\csname PY@tok@si\endcsname{\let\PY@bf=\textbf\def\PY@tc##1{\textcolor[rgb]{0.73,0.40,0.53}{##1}}}
\expandafter\def\csname PY@tok@se\endcsname{\let\PY@bf=\textbf\def\PY@tc##1{\textcolor[rgb]{0.73,0.40,0.13}{##1}}}
\expandafter\def\csname PY@tok@sr\endcsname{\def\PY@tc##1{\textcolor[rgb]{0.73,0.40,0.53}{##1}}}
\expandafter\def\csname PY@tok@ss\endcsname{\def\PY@tc##1{\textcolor[rgb]{0.10,0.09,0.49}{##1}}}
\expandafter\def\csname PY@tok@sx\endcsname{\def\PY@tc##1{\textcolor[rgb]{0.00,0.50,0.00}{##1}}}
\expandafter\def\csname PY@tok@m\endcsname{\def\PY@tc##1{\textcolor[rgb]{0.40,0.40,0.40}{##1}}}
\expandafter\def\csname PY@tok@gh\endcsname{\let\PY@bf=\textbf\def\PY@tc##1{\textcolor[rgb]{0.00,0.00,0.50}{##1}}}
\expandafter\def\csname PY@tok@gu\endcsname{\let\PY@bf=\textbf\def\PY@tc##1{\textcolor[rgb]{0.50,0.00,0.50}{##1}}}
\expandafter\def\csname PY@tok@gd\endcsname{\def\PY@tc##1{\textcolor[rgb]{0.63,0.00,0.00}{##1}}}
\expandafter\def\csname PY@tok@gi\endcsname{\def\PY@tc##1{\textcolor[rgb]{0.00,0.63,0.00}{##1}}}
\expandafter\def\csname PY@tok@gr\endcsname{\def\PY@tc##1{\textcolor[rgb]{1.00,0.00,0.00}{##1}}}
\expandafter\def\csname PY@tok@ge\endcsname{\let\PY@it=\textit}
\expandafter\def\csname PY@tok@gs\endcsname{\let\PY@bf=\textbf}
\expandafter\def\csname PY@tok@gp\endcsname{\let\PY@bf=\textbf\def\PY@tc##1{\textcolor[rgb]{0.00,0.00,0.50}{##1}}}
\expandafter\def\csname PY@tok@go\endcsname{\def\PY@tc##1{\textcolor[rgb]{0.53,0.53,0.53}{##1}}}
\expandafter\def\csname PY@tok@gt\endcsname{\def\PY@tc##1{\textcolor[rgb]{0.00,0.27,0.87}{##1}}}
\expandafter\def\csname PY@tok@err\endcsname{\def\PY@bc##1{\setlength{\fboxsep}{0pt}\fcolorbox[rgb]{1.00,0.00,0.00}{1,1,1}{\strut ##1}}}
\expandafter\def\csname PY@tok@kc\endcsname{\let\PY@bf=\textbf\def\PY@tc##1{\textcolor[rgb]{0.00,0.50,0.00}{##1}}}
\expandafter\def\csname PY@tok@kd\endcsname{\let\PY@bf=\textbf\def\PY@tc##1{\textcolor[rgb]{0.00,0.50,0.00}{##1}}}
\expandafter\def\csname PY@tok@kn\endcsname{\let\PY@bf=\textbf\def\PY@tc##1{\textcolor[rgb]{0.00,0.50,0.00}{##1}}}
\expandafter\def\csname PY@tok@kr\endcsname{\let\PY@bf=\textbf\def\PY@tc##1{\textcolor[rgb]{0.00,0.50,0.00}{##1}}}
\expandafter\def\csname PY@tok@bp\endcsname{\def\PY@tc##1{\textcolor[rgb]{0.00,0.50,0.00}{##1}}}
\expandafter\def\csname PY@tok@fm\endcsname{\def\PY@tc##1{\textcolor[rgb]{0.00,0.00,1.00}{##1}}}
\expandafter\def\csname PY@tok@vc\endcsname{\def\PY@tc##1{\textcolor[rgb]{0.10,0.09,0.49}{##1}}}
\expandafter\def\csname PY@tok@vg\endcsname{\def\PY@tc##1{\textcolor[rgb]{0.10,0.09,0.49}{##1}}}
\expandafter\def\csname PY@tok@vi\endcsname{\def\PY@tc##1{\textcolor[rgb]{0.10,0.09,0.49}{##1}}}
\expandafter\def\csname PY@tok@vm\endcsname{\def\PY@tc##1{\textcolor[rgb]{0.10,0.09,0.49}{##1}}}
\expandafter\def\csname PY@tok@sa\endcsname{\def\PY@tc##1{\textcolor[rgb]{0.73,0.13,0.13}{##1}}}
\expandafter\def\csname PY@tok@sb\endcsname{\def\PY@tc##1{\textcolor[rgb]{0.73,0.13,0.13}{##1}}}
\expandafter\def\csname PY@tok@sc\endcsname{\def\PY@tc##1{\textcolor[rgb]{0.73,0.13,0.13}{##1}}}
\expandafter\def\csname PY@tok@dl\endcsname{\def\PY@tc##1{\textcolor[rgb]{0.73,0.13,0.13}{##1}}}
\expandafter\def\csname PY@tok@s2\endcsname{\def\PY@tc##1{\textcolor[rgb]{0.73,0.13,0.13}{##1}}}
\expandafter\def\csname PY@tok@sh\endcsname{\def\PY@tc##1{\textcolor[rgb]{0.73,0.13,0.13}{##1}}}
\expandafter\def\csname PY@tok@s1\endcsname{\def\PY@tc##1{\textcolor[rgb]{0.73,0.13,0.13}{##1}}}
\expandafter\def\csname PY@tok@mb\endcsname{\def\PY@tc##1{\textcolor[rgb]{0.40,0.40,0.40}{##1}}}
\expandafter\def\csname PY@tok@mf\endcsname{\def\PY@tc##1{\textcolor[rgb]{0.40,0.40,0.40}{##1}}}
\expandafter\def\csname PY@tok@mh\endcsname{\def\PY@tc##1{\textcolor[rgb]{0.40,0.40,0.40}{##1}}}
\expandafter\def\csname PY@tok@mi\endcsname{\def\PY@tc##1{\textcolor[rgb]{0.40,0.40,0.40}{##1}}}
\expandafter\def\csname PY@tok@il\endcsname{\def\PY@tc##1{\textcolor[rgb]{0.40,0.40,0.40}{##1}}}
\expandafter\def\csname PY@tok@mo\endcsname{\def\PY@tc##1{\textcolor[rgb]{0.40,0.40,0.40}{##1}}}
\expandafter\def\csname PY@tok@ch\endcsname{\let\PY@it=\textit\def\PY@tc##1{\textcolor[rgb]{0.25,0.50,0.50}{##1}}}
\expandafter\def\csname PY@tok@cm\endcsname{\let\PY@it=\textit\def\PY@tc##1{\textcolor[rgb]{0.25,0.50,0.50}{##1}}}
\expandafter\def\csname PY@tok@cpf\endcsname{\let\PY@it=\textit\def\PY@tc##1{\textcolor[rgb]{0.25,0.50,0.50}{##1}}}
\expandafter\def\csname PY@tok@c1\endcsname{\let\PY@it=\textit\def\PY@tc##1{\textcolor[rgb]{0.25,0.50,0.50}{##1}}}
\expandafter\def\csname PY@tok@cs\endcsname{\let\PY@it=\textit\def\PY@tc##1{\textcolor[rgb]{0.25,0.50,0.50}{##1}}}

\def\PYZbs{\char`\\}
\def\PYZus{\char`\_}
\def\PYZob{\char`\{}
\def\PYZcb{\char`\}}
\def\PYZca{\char`\^}
\def\PYZam{\char`\&}
\def\PYZlt{\char`\<}
\def\PYZgt{\char`\>}
\def\PYZsh{\char`\#}
\def\PYZpc{\char`\%}
\def\PYZdl{\char`\$}
\def\PYZhy{\char`\-}
\def\PYZsq{\char`\'}
\def\PYZdq{\char`\"}
\def\PYZti{\char`\~}
% for compatibility with earlier versions
\def\PYZat{@}
\def\PYZlb{[}
\def\PYZrb{]}
\makeatother


    % Exact colors from NB
    \definecolor{incolor}{rgb}{0.0, 0.0, 0.5}
    \definecolor{outcolor}{rgb}{0.545, 0.0, 0.0}



    
    % Prevent overflowing lines due to hard-to-break entities
    \sloppy 
    % Setup hyperref package
    \hypersetup{
      breaklinks=true,  % so long urls are correctly broken across lines
      colorlinks=true,
      urlcolor=urlcolor,
      linkcolor=linkcolor,
      citecolor=citecolor,
      }
    % Slightly bigger margins than the latex defaults
    
    \geometry{verbose,tmargin=1in,bmargin=1in,lmargin=1in,rmargin=1in}
    
    

    \begin{document}
    
    
    \maketitle
    
    

    
    \hypertarget{geopandas-advanced-topics}{%
\section{GeoPandas: Advanced topics}\label{geopandas-advanced-topics}}

\href{https://github.com/emiliom/}{Emilio Mayorga, University of
Washington}. 2018-9-9

    \hypertarget{introduction}{%
\subsection{1. Introduction}\label{introduction}}

    We covered the basics of GeoPandas in the previous episode and notebook.
Here, we'll extend that introduction to illustrate additional aspects of
GeoPandas and its interactions with other Python libraries, covering
fancier mapping, analysis (unitary and binary spatial operators), raster
zonal stats + GeoPandas.

\textbf{Here are the main sections in this episode / notebook:} - Read
HydroBASINS for Western Washington from a PostGIS / PostgreSQL
relational database on the cloud - Dissolve into larger watersheds, and
reproject - Plot \texttt{choropleth} map based on calculated watershed
areas - Choropleth map as an interactive map with folium - Spatial join,
\texttt{sjoin}, of polygons on points - rasterstats: ``zonal''
statistics from polygons on rasters

    \hypertarget{set-up-packages-and-data-file-path}{%
\subsection{2. Set up packages and data file
path}\label{set-up-packages-and-data-file-path}}

We'll use these throughout the rest of the tutorial.

    \begin{Verbatim}[commandchars=\\\{\}]
{\color{incolor}In [{\color{incolor}1}]:} \PY{o}{\PYZpc{}}\PY{k}{matplotlib} inline
        
        \PY{k+kn}{from} \PY{n+nn}{\PYZus{}\PYZus{}future\PYZus{}\PYZus{}} \PY{k}{import} \PY{p}{(}\PY{n}{absolute\PYZus{}import}\PY{p}{,} \PY{n}{division}\PY{p}{,} \PY{n}{print\PYZus{}function}\PY{p}{)}
        \PY{k+kn}{import} \PY{n+nn}{os}
        \PY{k+kn}{import} \PY{n+nn}{json}
        \PY{k+kn}{import} \PY{n+nn}{psycopg2}
        
        \PY{k+kn}{import} \PY{n+nn}{matplotlib} \PY{k}{as} \PY{n+nn}{mpl}
        \PY{k+kn}{import} \PY{n+nn}{matplotlib}\PY{n+nn}{.}\PY{n+nn}{pyplot} \PY{k}{as} \PY{n+nn}{plt}
        
        \PY{k+kn}{from} \PY{n+nn}{shapely}\PY{n+nn}{.}\PY{n+nn}{geometry} \PY{k}{import} \PY{n}{Point}
        \PY{k+kn}{import} \PY{n+nn}{pandas} \PY{k}{as} \PY{n+nn}{pd}
        \PY{k+kn}{import} \PY{n+nn}{geopandas} \PY{k}{as} \PY{n+nn}{gpd}
        \PY{k+kn}{from} \PY{n+nn}{geopandas} \PY{k}{import} \PY{n}{GeoSeries}\PY{p}{,} \PY{n}{GeoDataFrame}
        
        \PY{n}{data\PYZus{}pth} \PY{o}{=} \PY{l+s+s2}{\PYZdq{}}\PY{l+s+s2}{../data}\PY{l+s+s2}{\PYZdq{}}
\end{Verbatim}


    \begin{Verbatim}[commandchars=\\\{\}]
{\color{incolor}In [{\color{incolor}2}]:} \PY{n}{mpl}\PY{o}{.}\PY{n}{\PYZus{}\PYZus{}version\PYZus{}\PYZus{}}\PY{p}{,} \PY{n}{pd}\PY{o}{.}\PY{n}{\PYZus{}\PYZus{}version\PYZus{}\PYZus{}}\PY{p}{,} \PY{n}{gpd}\PY{o}{.}\PY{n}{\PYZus{}\PYZus{}version\PYZus{}\PYZus{}}
\end{Verbatim}


\begin{Verbatim}[commandchars=\\\{\}]
{\color{outcolor}Out[{\color{outcolor}2}]:} ('2.2.2', '0.23.4', '0.4.0')
\end{Verbatim}
            
    \begin{Verbatim}[commandchars=\\\{\}]
{\color{incolor}In [{\color{incolor}3}]:} \PY{k}{with} \PY{n+nb}{open}\PY{p}{(}\PY{n}{os}\PY{o}{.}\PY{n}{path}\PY{o}{.}\PY{n}{join}\PY{p}{(}\PY{n}{data\PYZus{}pth}\PY{p}{,} \PY{l+s+s2}{\PYZdq{}}\PY{l+s+s2}{db.json}\PY{l+s+s2}{\PYZdq{}}\PY{p}{)}\PY{p}{)} \PY{k}{as} \PY{n}{f}\PY{p}{:}
            \PY{n}{db\PYZus{}conn\PYZus{}dict} \PY{o}{=} \PY{n}{json}\PY{o}{.}\PY{n}{load}\PY{p}{(}\PY{n}{f}\PY{p}{)}
\end{Verbatim}


    \hypertarget{read-hydrobasins-north-america-dataset-extracting-western-washington}{%
\subsection{3. Read HydroBASINS North America dataset, extracting
Western
Washington}\label{read-hydrobasins-north-america-dataset-extracting-western-washington}}

Read \href{http://hydrosheds.org/page/hydrobasins}{HydroBASINS}
``all-levels'' (lev00) hierarchical watersheds dataset for North America
and the Caribbean (\texttt{hybas\_na\_lev00\_v1c}), from Amazon Cloud
PostgreSQL/PostGIS database. \textbf{Watersheds in the dataset are at
the finest (highest resolution) ``Pfastetter'' hierarchical level, level
12.} HydroBASINS dataset technical documentation is
\href{http://hydrosheds.org/images/inpages/HydroBASINS_TechDoc_v1c.pdf}{here}.

\texttt{read\_postgis} is called as before, except now we'll apply a SQL
filter (server side) to the PostGIS dataset to select only the
Pfastetter level-4 watershed with code 7831:
\texttt{WHERE\ pfaf\_4\ =\ 7831}. This is \textbf{most of Western
Washington.} Watershed polygons will still be read at their original
level 12 resolution.

For a more in-depth look at interacting with spatial relational
databases, see the eScience Institute tutorial
\href{https://uwescience.github.io/SQL-geospatial-tutorial/}{Introduction
to SQL and Geospatial Data Processing}

    \begin{Verbatim}[commandchars=\\\{\}]
{\color{incolor}In [{\color{incolor}4}]:} \PY{n}{conn} \PY{o}{=} \PY{n}{psycopg2}\PY{o}{.}\PY{n}{connect}\PY{p}{(}\PY{o}{*}\PY{o}{*}\PY{n}{db\PYZus{}conn\PYZus{}dict}\PY{p}{)}
\end{Verbatim}


    \begin{Verbatim}[commandchars=\\\{\}]
{\color{incolor}In [{\color{incolor}5}]:} \PY{n}{hydrobas\PYZus{}ww} \PY{o}{=} \PY{n}{gpd}\PY{o}{.}\PY{n}{read\PYZus{}postgis}\PY{p}{(}
            \PY{l+s+s2}{\PYZdq{}}\PY{l+s+s2}{SELECT * FROM hybas\PYZus{}na\PYZus{}lev00\PYZus{}v1c WHERE pfaf\PYZus{}4 = 7831}\PY{l+s+s2}{\PYZdq{}}\PY{p}{,} \PY{n}{conn}\PY{p}{,} 
            \PY{n}{geom\PYZus{}col}\PY{o}{=}\PY{l+s+s1}{\PYZsq{}}\PY{l+s+s1}{polygongeom}\PY{l+s+s1}{\PYZsq{}}\PY{p}{,}
            \PY{n}{coerce\PYZus{}float}\PY{o}{=}\PY{k+kc}{False}\PY{p}{)}
\end{Verbatim}


    \begin{Verbatim}[commandchars=\\\{\}]
{\color{incolor}In [{\color{incolor}6}]:} \PY{n}{conn}\PY{o}{.}\PY{n}{close}\PY{p}{(}\PY{p}{)}
\end{Verbatim}


    \begin{Verbatim}[commandchars=\\\{\}]
{\color{incolor}In [{\color{incolor}7}]:} \PY{n}{hydrobas\PYZus{}ww}\PY{o}{.}\PY{n}{crs}
\end{Verbatim}


\begin{Verbatim}[commandchars=\\\{\}]
{\color{outcolor}Out[{\color{outcolor}7}]:} \{'init': 'epsg:4326'\}
\end{Verbatim}
            
    \begin{Verbatim}[commandchars=\\\{\}]
{\color{incolor}In [{\color{incolor}8}]:} \PY{n+nb}{len}\PY{p}{(}\PY{n}{hydrobas\PYZus{}ww}\PY{p}{)}
\end{Verbatim}


\begin{Verbatim}[commandchars=\\\{\}]
{\color{outcolor}Out[{\color{outcolor}8}]:} 413
\end{Verbatim}
            
    413 polygon features returned. Let's examine the attributes available,
using the first feature as an example.

    \begin{Verbatim}[commandchars=\\\{\}]
{\color{incolor}In [{\color{incolor}9}]:} \PY{n}{hydrobas\PYZus{}ww}\PY{o}{.}\PY{n}{iloc}\PY{p}{[}\PY{l+m+mi}{0}\PY{p}{]}
\end{Verbatim}


\begin{Verbatim}[commandchars=\\\{\}]
{\color{outcolor}Out[{\color{outcolor}9}]:} gid                                                        19945
        hybas\_id                                             7.00001e+09
        next\_down                                                      0
        next\_sink                                            7.00001e+09
        main\_bas                                             7.00001e+09
        dist\_sink                                                      0
        dist\_main                                                      0
        sub\_area                                                   135.4
        up\_area                                                    135.4
        endo                                                           0
        coast                                                          0
        order                                                          1
        sort                                                       19945
        pfaf\_1                                                         7
        pfaf\_2                                                        78
        pfaf\_3                                                       783
        pfaf\_4                                                      7831
        pfaf\_5                                                     78310
        pfaf\_6                                                    783101
        pfaf\_7                                                   7831010
        pfaf\_8                                                  78310101
        pfaf\_9                                                 783101010
        pfaf\_10                                              7.83101e+09
        pfaf\_11                                              7.83101e+10
        pfaf\_12                                              7.83101e+11
        polygongeom    (POLYGON ((-123.35 46.36666666666669, -123.349{\ldots}
        Name: 0, dtype: object
\end{Verbatim}
            
    \textbf{Plot a categorical map with coloring based on the aggregating
column \texttt{pfaf\_7}.} Watershed \emph{boundaries} are at the
high-resolution Pfastetter level 12.\\
Note: pick a color map (\texttt{cmap}) appropriate for your data.
\href{https://matplotlib.org/tutorials/colors/colormaps.html}{Get to
know the matplotlib color maps.}

    \begin{Verbatim}[commandchars=\\\{\}]
{\color{incolor}In [{\color{incolor}10}]:} \PY{n}{hydrobas\PYZus{}ww}\PY{o}{.}\PY{n}{plot}\PY{p}{(}\PY{n}{column}\PY{o}{=}\PY{l+s+s1}{\PYZsq{}}\PY{l+s+s1}{pfaf\PYZus{}7}\PY{l+s+s1}{\PYZsq{}}\PY{p}{,} \PY{n}{cmap}\PY{o}{=}\PY{l+s+s1}{\PYZsq{}}\PY{l+s+s1}{tab20}\PY{l+s+s1}{\PYZsq{}}\PY{p}{,} \PY{n}{categorical}\PY{o}{=}\PY{k+kc}{True}\PY{p}{,} \PY{n}{figsize}\PY{o}{=}\PY{p}{(}\PY{l+m+mi}{14}\PY{p}{,} \PY{l+m+mi}{8}\PY{p}{)}\PY{p}{)}\PY{p}{;}
\end{Verbatim}


    \begin{center}
    \adjustimage{max size={0.9\linewidth}{0.9\paperheight}}{output_16_0.png}
    \end{center}
    { \hspace*{\fill} \\}
    
    \hypertarget{rename-the-geodataframe-geometry-column-from-polygongeom-to-geometry-to-avoid-issues-with-other-packages}{%
\subsubsection{\texorpdfstring{Rename the GeoDataFrame geometry column
from \texttt{polygongeom} to \texttt{geometry} to avoid issues with
other
packages}{Rename the GeoDataFrame geometry column from polygongeom to geometry to avoid issues with other packages}}\label{rename-the-geodataframe-geometry-column-from-polygongeom-to-geometry-to-avoid-issues-with-other-packages}}

Unfortunately, \texttt{folium} choropleth and \texttt{rasterstats}
(demonstrated below) require the geometry column to be named
``geometry''. So, we'll rename it here first.

    \begin{Verbatim}[commandchars=\\\{\}]
{\color{incolor}In [{\color{incolor}11}]:} \PY{n}{hydrobas\PYZus{}ww} \PY{o}{=} \PY{n}{hydrobas\PYZus{}ww}\PY{o}{.}\PY{n}{rename}\PY{p}{(}\PY{n}{columns}\PY{o}{=}\PY{p}{\PYZob{}}\PY{l+s+s1}{\PYZsq{}}\PY{l+s+s1}{polygongeom}\PY{l+s+s1}{\PYZsq{}}\PY{p}{:} \PY{l+s+s1}{\PYZsq{}}\PY{l+s+s1}{geometry}\PY{l+s+s1}{\PYZsq{}}\PY{p}{\PYZcb{}}\PY{p}{)}
         \PY{n}{hydrobas\PYZus{}ww}\PY{o}{.}\PY{n}{\PYZus{}geometry\PYZus{}column\PYZus{}name} \PY{o}{=} \PY{l+s+s1}{\PYZsq{}}\PY{l+s+s1}{geometry}\PY{l+s+s1}{\PYZsq{}}
\end{Verbatim}


    \hypertarget{dissolve-into-larger-watersheds-and-reproject}{%
\subsection{4. Dissolve into larger watersheds, and
reproject}\label{dissolve-into-larger-watersheds-and-reproject}}

    \hypertarget{dissolve-source-polygons-into-larger-watersheds-based-on-attribute-values}{%
\subsubsection{Dissolve source polygons into larger watersheds based on
attribute
values}\label{dissolve-source-polygons-into-larger-watersheds-based-on-attribute-values}}

Apply GeoDataFrame
\href{http://geopandas.org/aggregation_with_dissolve.html}{dissolve}
aggregation method (implemented from lower level \texttt{shapely}
operators) on level-7 Pfastetter codes (\texttt{pfaf\_7}) shown in the
plot above. Aggregate attributes, retaining only \texttt{pfaf\_7} and
\texttt{pfaf\_6} (plus \texttt{geometry}, of course). This operation
results in only 17 polygons, from the original 413.

    \begin{Verbatim}[commandchars=\\\{\}]
{\color{incolor}In [{\color{incolor}12}]:} \PY{n}{cols} \PY{o}{=} \PY{p}{[}\PY{l+s+s1}{\PYZsq{}}\PY{l+s+s1}{pfaf\PYZus{}6}\PY{l+s+s1}{\PYZsq{}}\PY{p}{,} \PY{l+s+s1}{\PYZsq{}}\PY{l+s+s1}{pfaf\PYZus{}7}\PY{l+s+s1}{\PYZsq{}}\PY{p}{,} \PY{l+s+s1}{\PYZsq{}}\PY{l+s+s1}{geometry}\PY{l+s+s1}{\PYZsq{}}\PY{p}{]}
         \PY{n}{hydrobas\PYZus{}ww\PYZus{}p7} \PY{o}{=} \PY{n}{hydrobas\PYZus{}ww}\PY{p}{[}\PY{n}{cols}\PY{p}{]}\PY{o}{.}\PY{n}{dissolve}\PY{p}{(}\PY{n}{by}\PY{o}{=}\PY{l+s+s1}{\PYZsq{}}\PY{l+s+s1}{pfaf\PYZus{}7}\PY{l+s+s1}{\PYZsq{}}\PY{p}{,} \PY{n}{aggfunc}\PY{o}{=}\PY{l+s+s1}{\PYZsq{}}\PY{l+s+s1}{first}\PY{l+s+s1}{\PYZsq{}}\PY{p}{,} \PY{n}{as\PYZus{}index}\PY{o}{=}\PY{k+kc}{False}\PY{p}{)}
         \PY{n+nb}{len}\PY{p}{(}\PY{n}{hydrobas\PYZus{}ww\PYZus{}p7}\PY{p}{)}
\end{Verbatim}


\begin{Verbatim}[commandchars=\\\{\}]
{\color{outcolor}Out[{\color{outcolor}12}]:} 17
\end{Verbatim}
            
    Let's examine some of the features.

    \begin{Verbatim}[commandchars=\\\{\}]
{\color{incolor}In [{\color{incolor}13}]:} \PY{n}{hydrobas\PYZus{}ww\PYZus{}p7}\PY{o}{.}\PY{n}{head}\PY{p}{(}\PY{p}{)}
\end{Verbatim}


\begin{Verbatim}[commandchars=\\\{\}]
{\color{outcolor}Out[{\color{outcolor}13}]:}     pfaf\_7                                           geometry  pfaf\_6
         0  7831010  (POLYGON ((-123.4666666666666 46.2666666666666{\ldots}  783101
         1  7831020  POLYGON ((-123.1791666666666 46.33333333333336{\ldots}  783102
         2  7831031  (POLYGON ((-123.9597222222222 46.9666666666667{\ldots}  783103
         3  7831032  POLYGON ((-123.8583333333333 47.39583333333336{\ldots}  783103
         4  7831033  POLYGON ((-124.3 47.34583333333336, -124.30221{\ldots}  783103
\end{Verbatim}
            
    Plot the results. Looks like the previous plot, except the polygon
boundaries are now the pfaf\_7 watersheds.

    \begin{Verbatim}[commandchars=\\\{\}]
{\color{incolor}In [{\color{incolor}14}]:} \PY{n}{hydrobas\PYZus{}ww\PYZus{}p7}\PY{o}{.}\PY{n}{plot}\PY{p}{(}\PY{n}{column}\PY{o}{=}\PY{l+s+s1}{\PYZsq{}}\PY{l+s+s1}{pfaf\PYZus{}7}\PY{l+s+s1}{\PYZsq{}}\PY{p}{,} \PY{n}{cmap}\PY{o}{=}\PY{l+s+s1}{\PYZsq{}}\PY{l+s+s1}{tab20}\PY{l+s+s1}{\PYZsq{}}\PY{p}{,} \PY{n}{categorical}\PY{o}{=}\PY{k+kc}{True}\PY{p}{,} \PY{n}{edgecolor}\PY{o}{=}\PY{l+s+s1}{\PYZsq{}}\PY{l+s+s1}{white}\PY{l+s+s1}{\PYZsq{}}\PY{p}{,}
                             \PY{n}{figsize}\PY{o}{=}\PY{p}{(}\PY{l+m+mi}{14}\PY{p}{,} \PY{l+m+mi}{8}\PY{p}{)}\PY{p}{)}\PY{p}{;}
\end{Verbatim}


    \begin{center}
    \adjustimage{max size={0.9\linewidth}{0.9\paperheight}}{output_25_0.png}
    \end{center}
    { \hspace*{\fill} \\}
    
    \textbf{\emph{NOTE/WATCH:}}\\
\textbf{Beware that \texttt{dissolve} may fail if there are ``invalid''
geometries.} This code is based on a GeoDataFrame examined in the
previous, intro notebook. The 6 geometries/points reported are invalid
(and are reported by the \texttt{is\_valid()} method). This dissolve
statement does work, though.

\begin{Shaded}
\begin{Highlighting}[]
\NormalTok{seas_grp }\OperatorTok{=}\NormalTok{ seas[[}\StringTok{'oceans'}\NormalTok{, }\StringTok{'geometry'}\NormalTok{]]}
\NormalTok{seas_oceans_diss }\OperatorTok{=}\NormalTok{ seas_grp[seas_grp.geometry.is_valid].dissolve(by}\OperatorTok{=}\StringTok{'oceans'}\NormalTok{)}

\NormalTok{Ring Self}\OperatorTok{-}\NormalTok{intersection at }\KeywordTok{or}\NormalTok{ near point }\FloatTok{10.407218181818182} \FloatTok{54.821390909090908}
\NormalTok{Self}\OperatorTok{-}\NormalTok{intersection at }\KeywordTok{or}\NormalTok{ near point }\FloatTok{-79.365827272727287} \FloatTok{76.296645454545455}
\NormalTok{Ring Self}\OperatorTok{-}\NormalTok{intersection at }\KeywordTok{or}\NormalTok{ near point }\FloatTok{10.979445510225332} \FloatTok{54.380555030408686}
\NormalTok{Ring Self}\OperatorTok{-}\NormalTok{intersection at }\KeywordTok{or}\NormalTok{ near point }\FloatTok{133.61550925464189} \FloatTok{-4.3005540903175188}
\NormalTok{Ring Self}\OperatorTok{-}\NormalTok{intersection at }\KeywordTok{or}\NormalTok{ near point }\FloatTok{121.91067196634913} \FloatTok{-5.0593090510592447}
\NormalTok{Ring Self}\OperatorTok{-}\NormalTok{intersection at }\KeywordTok{or}\NormalTok{ near point }\FloatTok{115.29553592754269} \FloatTok{-7.0082630551828515}
\end{Highlighting}
\end{Shaded}

    \hypertarget{reproject-transform-to-wa-state-plane-south-epsg2927}{%
\subsubsection{Reproject (transform) to WA State Plane South,
epsg:2927}\label{reproject-transform-to-wa-state-plane-south-epsg2927}}

Partly so we can calculate polygon areas in linear units, not geodetic
degrees. But also because that's the projection used by most state and
local governments in Washington. - http://epsg.io/2927 -
http://spatialreference.org/ref/epsg/2927/ -
\href{http://www.epsg-registry.org/report.htm?type=selection\&entity=urn:ogc:def:crs:EPSG::2927\&reportDetail=short\&style=urn:uuid:report-style:default-with-code\&style_name=OGP\%20Default\%20With\%20Code\&title=EPSG:2927}{Report
from http://www.epsg-registry.org}

    No need to go to a web site to learn more about what \texttt{epsg:2927}
is. Use \texttt{pyepsg}, which issues queries to http://epsg.io web
services.

    \begin{Verbatim}[commandchars=\\\{\}]
{\color{incolor}In [{\color{incolor}15}]:} \PY{k+kn}{import} \PY{n+nn}{pyepsg}
\end{Verbatim}


    Extract the epsg code from the string returned by
\texttt{crs{[}\textquotesingle{}init\textquotesingle{}{]}}, then query
epsg \texttt{2927}.

    \begin{Verbatim}[commandchars=\\\{\}]
{\color{incolor}In [{\color{incolor}16}]:} \PY{n}{hydrobas\PYZus{}ww\PYZus{}p7\PYZus{}epsg\PYZus{}str} \PY{o}{=} \PY{n}{hydrobas\PYZus{}ww\PYZus{}p7}\PY{o}{.}\PY{n}{crs}\PY{p}{[}\PY{l+s+s1}{\PYZsq{}}\PY{l+s+s1}{init}\PY{l+s+s1}{\PYZsq{}}\PY{p}{]}\PY{o}{.}\PY{n}{split}\PY{p}{(}\PY{l+s+s1}{\PYZsq{}}\PY{l+s+s1}{:}\PY{l+s+s1}{\PYZsq{}}\PY{p}{)}\PY{p}{[}\PY{l+m+mi}{1}\PY{p}{]}
         \PY{n}{pyepsg}\PY{o}{.}\PY{n}{get}\PY{p}{(}\PY{n}{hydrobas\PYZus{}ww\PYZus{}p7\PYZus{}epsg\PYZus{}str}\PY{p}{)}
\end{Verbatim}


\begin{Verbatim}[commandchars=\\\{\}]
{\color{outcolor}Out[{\color{outcolor}16}]:} <GeodeticCRS: 4326, WGS 84>
\end{Verbatim}
            
    \begin{Verbatim}[commandchars=\\\{\}]
{\color{incolor}In [{\color{incolor}17}]:} \PY{n}{pyepsg}\PY{o}{.}\PY{n}{get}\PY{p}{(}\PY{l+s+s1}{\PYZsq{}}\PY{l+s+s1}{2927}\PY{l+s+s1}{\PYZsq{}}\PY{p}{)}
\end{Verbatim}


\begin{Verbatim}[commandchars=\\\{\}]
{\color{outcolor}Out[{\color{outcolor}17}]:} <ProjectedCRS: 2927, NAD83(HARN) / Washington South (ftUS)>
\end{Verbatim}
            
    \textbf{Apply the crs transformation (reprojection)} using
\texttt{to\_crs} method.

    \begin{Verbatim}[commandchars=\\\{\}]
{\color{incolor}In [{\color{incolor}19}]:} \PY{n}{hydrobas\PYZus{}ww\PYZus{}p7\PYZus{}wasp} \PY{o}{=} \PY{n}{hydrobas\PYZus{}ww\PYZus{}p7}\PY{o}{.}\PY{n}{to\PYZus{}crs}\PY{p}{(}\PY{n}{epsg}\PY{o}{=}\PY{l+m+mi}{2927}\PY{p}{)}
\end{Verbatim}


    \textbf{Plot the reprojected map.} Note that, being in a planar project
(not geodetic), the shape looks different compared to the previous map.
More ``normal''. And the axes are now in \texttt{feet} relative to some
origin.

    \begin{Verbatim}[commandchars=\\\{\}]
{\color{incolor}In [{\color{incolor}20}]:} \PY{n}{hydrobas\PYZus{}ww\PYZus{}p7\PYZus{}wasp}\PY{o}{.}\PY{n}{plot}\PY{p}{(}\PY{n}{column}\PY{o}{=}\PY{l+s+s1}{\PYZsq{}}\PY{l+s+s1}{pfaf\PYZus{}7}\PY{l+s+s1}{\PYZsq{}}\PY{p}{,} \PY{n}{cmap}\PY{o}{=}\PY{l+s+s1}{\PYZsq{}}\PY{l+s+s1}{tab20}\PY{l+s+s1}{\PYZsq{}}\PY{p}{,} \PY{n}{categorical}\PY{o}{=}\PY{k+kc}{True}\PY{p}{,} \PY{n}{edgecolor}\PY{o}{=}\PY{l+s+s1}{\PYZsq{}}\PY{l+s+s1}{white}\PY{l+s+s1}{\PYZsq{}}\PY{p}{,}
                                  \PY{n}{figsize}\PY{o}{=}\PY{p}{(}\PY{l+m+mi}{14}\PY{p}{,} \PY{l+m+mi}{8}\PY{p}{)}\PY{p}{)}\PY{p}{;}
\end{Verbatim}


    \begin{center}
    \adjustimage{max size={0.9\linewidth}{0.9\paperheight}}{output_36_0.png}
    \end{center}
    { \hspace*{\fill} \\}
    
    \hypertarget{plot-choropleth-map-based-on-calculated-watershed-areas}{%
\subsection{\texorpdfstring{5. Plot \texttt{choropleth} map based on
calculated watershed
areas}{5. Plot choropleth map based on calculated watershed areas}}\label{plot-choropleth-map-based-on-calculated-watershed-areas}}

As the projection is in \texttt{feet}, auto-calculated polygon areas
will be in feet2. So let's convert to miles2 first (why not!). We'll add
a new column to the GeoDataFrame.

    \begin{Verbatim}[commandchars=\\\{\}]
{\color{incolor}In [{\color{incolor}21}]:} \PY{n}{hydrobas\PYZus{}ww\PYZus{}p7\PYZus{}wasp}\PY{p}{[}\PY{l+s+s1}{\PYZsq{}}\PY{l+s+s1}{area\PYZus{}mi2}\PY{l+s+s1}{\PYZsq{}}\PY{p}{]} \PY{o}{=} \PY{n}{hydrobas\PYZus{}ww\PYZus{}p7\PYZus{}wasp}\PY{o}{.}\PY{n}{geometry}\PY{o}{.}\PY{n}{area} \PY{o}{/} \PY{l+m+mi}{27878400}
         \PY{n}{hydrobas\PYZus{}ww\PYZus{}p7\PYZus{}wasp}\PY{o}{.}\PY{n}{head}\PY{p}{(}\PY{l+m+mi}{3}\PY{p}{)}
\end{Verbatim}


\begin{Verbatim}[commandchars=\\\{\}]
{\color{outcolor}Out[{\color{outcolor}21}]:}     pfaf\_7                                           geometry  pfaf\_6  \textbackslash{}
         0  7831010  (POLYGON ((890315.2572612347 354459.8899780738{\ldots}  783101   
         1  7831020  POLYGON ((963803.8027083561 376154.9965688475,{\ldots}  783102   
         2  7831031  (POLYGON ((776917.1877152125 614568.383332147,{\ldots}  783103   
         
               area\_mi2  
         0  1375.137396  
         1  2107.945774  
         2   528.472846  
\end{Verbatim}
            
    \textbf{\emph{NOTE/FUN:}}\\
Now you could get the area of a pfaf\_6 watershed via simple Pandas
DataFrame \texttt{groupby} aggregation (sum).

    Plot the choloropleth, using \texttt{area\_mi2}.

The ``fisher\_jenks'' value segmentation \texttt{scheme} (using 7
segments, k=7) used is one of the available
\texttt{pysal.esda.mapclassify.Map\_Classifier} classifiers from the
powerful \href{http://pysal.org/}{PySAL package} (Python Spatial
Analysis Library); GeoPandas can use these classifiers if PySAL is
installed, as it is here. To get the list of classifiers, use:

\begin{Shaded}
\begin{Highlighting}[]
\ImportTok{import}\NormalTok{ pysal}
\BuiltInTok{print}\NormalTok{(pysal.esda.mapclassify.Map_Classifier.__doc__)}
\end{Highlighting}
\end{Shaded}

    \begin{Verbatim}[commandchars=\\\{\}]
{\color{incolor}In [{\color{incolor}22}]:} \PY{n}{f}\PY{p}{,} \PY{n}{ax} \PY{o}{=} \PY{n}{plt}\PY{o}{.}\PY{n}{subplots}\PY{p}{(}\PY{l+m+mi}{1}\PY{p}{,} \PY{n}{figsize}\PY{o}{=}\PY{p}{(}\PY{l+m+mi}{10}\PY{p}{,} \PY{l+m+mi}{6}\PY{p}{)}\PY{p}{)}
         \PY{n}{ax}\PY{o}{.}\PY{n}{set\PYZus{}title}\PY{p}{(}\PY{l+s+s1}{\PYZsq{}}\PY{l+s+s1}{Watersheds by area (\PYZdl{}mi\PYZca{}2\PYZdl{})}\PY{l+s+s1}{\PYZsq{}}\PY{p}{)}
         \PY{n}{hydrobas\PYZus{}ww\PYZus{}p7\PYZus{}wasp}\PY{o}{.}\PY{n}{plot}\PY{p}{(}\PY{n}{column}\PY{o}{=}\PY{l+s+s1}{\PYZsq{}}\PY{l+s+s1}{area\PYZus{}mi2}\PY{l+s+s1}{\PYZsq{}}\PY{p}{,} \PY{n}{scheme}\PY{o}{=}\PY{l+s+s1}{\PYZsq{}}\PY{l+s+s1}{fisher\PYZus{}jenks}\PY{l+s+s1}{\PYZsq{}}\PY{p}{,} \PY{n}{k}\PY{o}{=}\PY{l+m+mi}{7}\PY{p}{,} 
                                  \PY{n}{cmap}\PY{o}{=}\PY{n}{plt}\PY{o}{.}\PY{n}{cm}\PY{o}{.}\PY{n}{Blues}\PY{p}{,} \PY{n}{legend}\PY{o}{=}\PY{k+kc}{True}\PY{p}{,} \PY{n}{ax}\PY{o}{=}\PY{n}{ax}\PY{p}{)}
         \PY{n}{ax}\PY{o}{.}\PY{n}{set\PYZus{}axis\PYZus{}off}\PY{p}{(}\PY{p}{)}
         \PY{n}{plt}\PY{o}{.}\PY{n}{axis}\PY{p}{(}\PY{l+s+s1}{\PYZsq{}}\PY{l+s+s1}{equal}\PY{l+s+s1}{\PYZsq{}}\PY{p}{)}\PY{p}{;}
\end{Verbatim}


    \begin{Verbatim}[commandchars=\\\{\}]
/opt/conda/lib/python3.6/site-packages/pysal/\_\_init\_\_.py:65: VisibleDeprecationWarning: PySAL's API will be changed on 2018-12-31. The last release made with this API is version 1.14.4. A preview of the next API version is provided in the `pysal` 2.0 prelease candidate. The API changes and a guide on how to change imports is provided at https://pysal.org/about
  ), VisibleDeprecationWarning)

    \end{Verbatim}

    \begin{center}
    \adjustimage{max size={0.9\linewidth}{0.9\paperheight}}{output_41_1.png}
    \end{center}
    { \hspace*{\fill} \\}
    
    \textbf{\emph{NOTE/FUN:}}\\
Let's stop for a bit to explore on your own, hack with your neighbors,
ask questions.

    \hypertarget{choropleth-map-as-an-interactive-map-with-folium}{%
\subsection{6. Choropleth map as an interactive map with
folium}\label{choropleth-map-as-an-interactive-map-with-folium}}

\href{https://github.com/python-visualization/folium}{Folium} is very
cool, specially for use in Jupyter notebooks; or to export into
stand-alone HTML.

    \begin{Verbatim}[commandchars=\\\{\}]
{\color{incolor}In [{\color{incolor}23}]:} \PY{k+kn}{import} \PY{n+nn}{folium}
\end{Verbatim}


    \begin{Verbatim}[commandchars=\\\{\}]
{\color{incolor}In [{\color{incolor}24}]:} \PY{n}{folium}\PY{o}{.}\PY{n}{\PYZus{}\PYZus{}version\PYZus{}\PYZus{}}
\end{Verbatim}


\begin{Verbatim}[commandchars=\\\{\}]
{\color{outcolor}Out[{\color{outcolor}24}]:} '0.6.0'
\end{Verbatim}
            
    \texttt{m.choropleth} internally splits the geometry from the other
attributes in \texttt{hydrobas\_ww\_p7\_wasp}, and rejoins them based on
the key \texttt{pfaf\_7}. \texttt{key\_on} uses an attribute reference
derived from GeoJSON representations; this is awkward, and hopefully
will be simplified in future folium implementations.

    \begin{Verbatim}[commandchars=\\\{\}]
{\color{incolor}In [{\color{incolor}25}]:} \PY{n}{hydrobas\PYZus{}ww\PYZus{}p7\PYZus{}wasp}\PY{o}{.}\PY{n}{head}\PY{p}{(}\PY{p}{)}
\end{Verbatim}


\begin{Verbatim}[commandchars=\\\{\}]
{\color{outcolor}Out[{\color{outcolor}25}]:}     pfaf\_7                                           geometry  pfaf\_6  \textbackslash{}
         0  7831010  (POLYGON ((890315.2572612347 354459.8899780738{\ldots}  783101   
         1  7831020  POLYGON ((963803.8027083561 376154.9965688475,{\ldots}  783102   
         2  7831031  (POLYGON ((776917.1877152125 614568.383332147,{\ldots}  783103   
         3  7831032  POLYGON ((808869.5755557655 769864.5311527567,{\ldots}  783103   
         4  7831033  POLYGON ((698711.1808079176 756609.8674803674,{\ldots}  783103   
         
               area\_mi2  
         0  1375.137396  
         1  2107.945774  
         2   528.472846  
         3   441.528065  
         4   106.456891  
\end{Verbatim}
            
    \begin{Verbatim}[commandchars=\\\{\}]
{\color{incolor}In [{\color{incolor}26}]:} \PY{n}{m} \PY{o}{=} \PY{n}{folium}\PY{o}{.}\PY{n}{Map}\PY{p}{(}\PY{n}{location}\PY{o}{=}\PY{p}{[}\PY{l+m+mf}{47.8}\PY{p}{,} \PY{o}{\PYZhy{}}\PY{l+m+mf}{122.5}\PY{p}{]}\PY{p}{,} \PY{n}{zoom\PYZus{}start}\PY{o}{=}\PY{l+m+mi}{7}\PY{p}{,} \PY{n}{tiles}\PY{o}{=}\PY{l+s+s2}{\PYZdq{}}\PY{l+s+s2}{cartodbpositron}\PY{l+s+s2}{\PYZdq{}}\PY{p}{)}
         
         \PY{n}{m}\PY{o}{.}\PY{n}{choropleth}\PY{p}{(}
             \PY{n}{geo\PYZus{}data}\PY{o}{=}\PY{n}{hydrobas\PYZus{}ww\PYZus{}p7\PYZus{}wasp}\PY{p}{,}
             \PY{n}{data}\PY{o}{=}\PY{n}{hydrobas\PYZus{}ww\PYZus{}p7\PYZus{}wasp}\PY{p}{,}
             \PY{n}{columns}\PY{o}{=}\PY{p}{[}\PY{l+s+s1}{\PYZsq{}}\PY{l+s+s1}{pfaf\PYZus{}7}\PY{l+s+s1}{\PYZsq{}}\PY{p}{,} \PY{l+s+s1}{\PYZsq{}}\PY{l+s+s1}{area\PYZus{}mi2}\PY{l+s+s1}{\PYZsq{}}\PY{p}{]}\PY{p}{,}
             \PY{n}{key\PYZus{}on}\PY{o}{=}\PY{l+s+s1}{\PYZsq{}}\PY{l+s+s1}{feature.properties.pfaf\PYZus{}7}\PY{l+s+s1}{\PYZsq{}}\PY{p}{,}
             \PY{n}{legend\PYZus{}name}\PY{o}{=}\PY{l+s+s1}{\PYZsq{}}\PY{l+s+s1}{Area (sq mi)}\PY{l+s+s1}{\PYZsq{}}\PY{p}{,} 
             \PY{n}{fill\PYZus{}color}\PY{o}{=}\PY{l+s+s1}{\PYZsq{}}\PY{l+s+s1}{YlGn}\PY{l+s+s1}{\PYZsq{}}\PY{p}{,}
             \PY{n}{fill\PYZus{}opacity}\PY{o}{=}\PY{l+m+mf}{0.4}\PY{p}{,}
             \PY{n}{highlight}\PY{o}{=}\PY{k+kc}{True}\PY{p}{)}
         
         \PY{n}{m}
\end{Verbatim}


\begin{Verbatim}[commandchars=\\\{\}]
{\color{outcolor}Out[{\color{outcolor}26}]:} <folium.folium.Map at 0x7f5819bc9208>
\end{Verbatim}
            
    This map is interactive, so play with it (zoom and pan). There is a lot
more to explore in Folium! This is just a teaser.

    \hypertarget{spatial-join-sjoin-of-polygons-on-points}{%
\subsection{\texorpdfstring{7. Spatial join, \texttt{sjoin}, of polygons
on
points}{7. Spatial join, sjoin, of polygons on points}}\label{spatial-join-sjoin-of-polygons-on-points}}

We'll use an old, local snapshot of NANOOS coastal and marine monitoring
stations in the Pacific NW, from the
\href{http://nvs.nanoos.org/Explorer}{NANOOS Visualization System (NVS)
Data Explorer}. While many stations are moorings on marine waters, some
are inshore or in tidal shores and will overlap the watershed
boundaries. The point file is in the
\href{http://www.geopackage.org}{GeoPackage} format, an OGC format
implemented in SQLite.

    \begin{Verbatim}[commandchars=\\\{\}]
{\color{incolor}In [{\color{incolor} }]:} \PY{n}{nanoosstations\PYZus{}gdf} \PY{o}{=} \PY{n}{gpd}\PY{o}{.}\PY{n}{read\PYZus{}file}\PY{p}{(}\PY{n}{os}\PY{o}{.}\PY{n}{path}\PY{o}{.}\PY{n}{join}\PY{p}{(}\PY{n}{data\PYZus{}pth}\PY{p}{,} \PY{l+s+s2}{\PYZdq{}}\PY{l+s+s2}{nanoos\PYZus{}nvs.gpkg}\PY{l+s+s2}{\PYZdq{}}\PY{p}{)}\PY{p}{)}
        \PY{n+nb}{len}\PY{p}{(}\PY{n}{nanoosstations\PYZus{}gdf}\PY{p}{)}
\end{Verbatim}


    \begin{Verbatim}[commandchars=\\\{\}]
{\color{incolor}In [{\color{incolor} }]:} \PY{n}{nanoosstations\PYZus{}gdf}\PY{o}{.}\PY{n}{iloc}\PY{p}{[}\PY{o}{\PYZhy{}}\PY{l+m+mi}{1}\PY{p}{]}
\end{Verbatim}


    Points are on the coasts of the Pacific NW (BC, WA, OR) and out in the
open ocean.

    \begin{Verbatim}[commandchars=\\\{\}]
{\color{incolor}In [{\color{incolor} }]:} \PY{n}{nanoosstations\PYZus{}gdf}\PY{o}{.}\PY{n}{plot}\PY{p}{(}\PY{n}{markersize}\PY{o}{=}\PY{l+m+mi}{15}\PY{p}{)}\PY{p}{;}
\end{Verbatim}


    \textbf{Apply ``inner'' spatial join with the \texttt{sjoin} operator}.
An inner join will retain only overlapping features. Then plot as a map
overlay on top of \texttt{hydrobas\_ww\_p7}, categorizing (coloring)
each point by the \texttt{pfaf\_6} watershed it's in.

    \begin{Verbatim}[commandchars=\\\{\}]
{\color{incolor}In [{\color{incolor} }]:} \PY{n}{nanoossta\PYZus{}hydrobas\PYZus{}ww\PYZus{}gdf} \PY{o}{=} \PY{n}{gpd}\PY{o}{.}\PY{n}{sjoin}\PY{p}{(}\PY{n}{nanoosstations\PYZus{}gdf}\PY{p}{,} \PY{n}{hydrobas\PYZus{}ww\PYZus{}p7}\PY{p}{,} \PY{n}{how}\PY{o}{=}\PY{l+s+s2}{\PYZdq{}}\PY{l+s+s2}{inner}\PY{l+s+s2}{\PYZdq{}}\PY{p}{)}
        \PY{n+nb}{len}\PY{p}{(}\PY{n}{nanoossta\PYZus{}hydrobas\PYZus{}ww\PYZus{}gdf}\PY{p}{)}
\end{Verbatim}


    \begin{Verbatim}[commandchars=\\\{\}]
{\color{incolor}In [{\color{incolor} }]:} \PY{n}{f}\PY{p}{,} \PY{n}{ax} \PY{o}{=} \PY{n}{plt}\PY{o}{.}\PY{n}{subplots}\PY{p}{(}\PY{l+m+mi}{1}\PY{p}{,} \PY{n}{figsize}\PY{o}{=}\PY{p}{(}\PY{l+m+mi}{10}\PY{p}{,} \PY{l+m+mi}{6}\PY{p}{)}\PY{p}{)}
        \PY{n}{ax}\PY{o}{.}\PY{n}{set\PYZus{}axis\PYZus{}off}\PY{p}{(}\PY{p}{)}
        \PY{n}{plt}\PY{o}{.}\PY{n}{axis}\PY{p}{(}\PY{l+s+s1}{\PYZsq{}}\PY{l+s+s1}{equal}\PY{l+s+s1}{\PYZsq{}}\PY{p}{)}
        \PY{n}{hydrobas\PYZus{}ww\PYZus{}p7}\PY{o}{.}\PY{n}{plot}\PY{p}{(}\PY{n}{ax}\PY{o}{=}\PY{n}{ax}\PY{p}{,} \PY{n}{cmap}\PY{o}{=}\PY{l+s+s1}{\PYZsq{}}\PY{l+s+s1}{Greys\PYZus{}r}\PY{l+s+s1}{\PYZsq{}}\PY{p}{,} \PY{n}{linewidth}\PY{o}{=}\PY{l+m+mf}{0.5}\PY{p}{,} \PY{n}{edgecolor}\PY{o}{=}\PY{l+s+s1}{\PYZsq{}}\PY{l+s+s1}{red}\PY{l+s+s1}{\PYZsq{}}\PY{p}{)}
        \PY{n}{nanoossta\PYZus{}hydrobas\PYZus{}ww\PYZus{}gdf}\PY{o}{.}\PY{n}{plot}\PY{p}{(}\PY{n}{column}\PY{o}{=}\PY{l+s+s1}{\PYZsq{}}\PY{l+s+s1}{pfaf\PYZus{}6}\PY{l+s+s1}{\PYZsq{}}\PY{p}{,} \PY{n}{markersize}\PY{o}{=}\PY{l+m+mi}{30}\PY{p}{,} 
                                       \PY{n}{categorical}\PY{o}{=}\PY{k+kc}{True}\PY{p}{,} \PY{n}{legend}\PY{o}{=}\PY{k+kc}{True}\PY{p}{,} \PY{n}{ax}\PY{o}{=}\PY{n}{ax}\PY{p}{)}\PY{p}{;}
\end{Verbatim}


    \hypertarget{rasterstats-zonal-statistics-from-polygons-on-rasters}{%
\subsection{8. rasterstats: ``zonal'' statistics from polygons on
rasters}\label{rasterstats-zonal-statistics-from-polygons-on-rasters}}

We'll end by mixing features from a GeoDataFrame with a raster, applying
zonal statistics using the cool and light weight
\href{https://github.com/perrygeo/python-rasterstats}{rasterstats}
package.

    Monthly Juy long-term climatology precipitation. The original monthly
time series data are from the \href{http://prism.oregonstate.edu}{PRISM
Climate Group}; the monthly climatology and Pacific NW clip were created
by your truly and Don Setiawan for the \href{http://bigcz.org}{BiGCZ
project}.

    \begin{Verbatim}[commandchars=\\\{\}]
{\color{incolor}In [{\color{incolor} }]:} \PY{n}{ppt\PYZus{}july\PYZus{}tif\PYZus{}pth} \PY{o}{=} \PY{n}{os}\PY{o}{.}\PY{n}{path}\PY{o}{.}\PY{n}{join}\PY{p}{(}\PY{n}{data\PYZus{}pth}\PY{p}{,} \PY{l+s+s1}{\PYZsq{}}\PY{l+s+s1}{prism\PYZus{}precipitation\PYZus{}july\PYZus{}climatology.tif}\PY{l+s+s1}{\PYZsq{}}\PY{p}{)}
\end{Verbatim}


    \hypertarget{rasterio}{%
\subsubsection{rasterio}\label{rasterio}}

    \texttt{rasterstas} uses
\href{https://mapbox.github.io/rasterio}{rasterio} to read rasters (and
\texttt{fiona} to read vector datasets), so we'll first do a quick
exploration of rasterio.

    \begin{Verbatim}[commandchars=\\\{\}]
{\color{incolor}In [{\color{incolor} }]:} \PY{k+kn}{import} \PY{n+nn}{rasterio}
        \PY{k+kn}{import} \PY{n+nn}{rasterio}\PY{n+nn}{.}\PY{n+nn}{plot} \PY{k}{as} \PY{n+nn}{rioplot}
\end{Verbatim}


    \begin{Verbatim}[commandchars=\\\{\}]
{\color{incolor}In [{\color{incolor} }]:} \PY{n}{rasterio}\PY{o}{.}\PY{n}{\PYZus{}\PYZus{}version\PYZus{}\PYZus{}}
\end{Verbatim}


    \begin{Verbatim}[commandchars=\\\{\}]
{\color{incolor}In [{\color{incolor} }]:} \PY{n}{ppt\PYZus{}july} \PY{o}{=} \PY{n}{rasterio}\PY{o}{.}\PY{n}{open}\PY{p}{(}\PY{n}{ppt\PYZus{}july\PYZus{}tif\PYZus{}pth}\PY{p}{)}
        \PY{n}{ppt\PYZus{}july}
\end{Verbatim}


    Examine the metadata read from the raster file (we can confirm CRS is
epsg:4326), then plot the raster.

    \begin{Verbatim}[commandchars=\\\{\}]
{\color{incolor}In [{\color{incolor} }]:} \PY{n}{ppt\PYZus{}july}\PY{o}{.}\PY{n}{meta}
\end{Verbatim}


    \begin{Verbatim}[commandchars=\\\{\}]
{\color{incolor}In [{\color{incolor} }]:} \PY{n}{rioplot}\PY{o}{.}\PY{n}{show}\PY{p}{(}\PY{n}{ppt\PYZus{}july}\PY{p}{,} \PY{n}{with\PYZus{}bounds}\PY{o}{=}\PY{k+kc}{True}\PY{p}{,} \PY{n}{cmap}\PY{o}{=}\PY{n}{plt}\PY{o}{.}\PY{n}{cm}\PY{o}{.}\PY{n}{Blues}\PY{p}{)}\PY{p}{;}
\end{Verbatim}


    \hypertarget{apply-rasterstas-zonal_stats}{%
\subsubsection{\texorpdfstring{Apply rasterstas
\texttt{zonal\_stats}}{Apply rasterstas zonal\_stats}}\label{apply-rasterstas-zonal_stats}}

    Apply \texttt{zonal\_stats} from \texttt{rasterstats} package. Can pass
a \texttt{GeoDataFrame} directly (instead of the file path to a GIS
file) because it implements our old friend, the
\texttt{\_\_geo\_interface\_\_} method. For the raster, we pass its file
path.

\texttt{zonal\_stats} returns a geojson with the original properties
plus the zonal statistics.

    \begin{Verbatim}[commandchars=\\\{\}]
{\color{incolor}In [{\color{incolor} }]:} \PY{k+kn}{import} \PY{n+nn}{rasterstats} \PY{k}{as} \PY{n+nn}{rs}
\end{Verbatim}


    \begin{Verbatim}[commandchars=\\\{\}]
{\color{incolor}In [{\color{incolor} }]:} \PY{n}{rs}\PY{o}{.}\PY{n}{\PYZus{}\PYZus{}version\PYZus{}\PYZus{}}
\end{Verbatim}


    \begin{Verbatim}[commandchars=\\\{\}]
{\color{incolor}In [{\color{incolor} }]:} \PY{n}{zonal\PYZus{}ppt\PYZus{}gjson} \PY{o}{=} \PY{n}{rs}\PY{o}{.}\PY{n}{zonal\PYZus{}stats}\PY{p}{(}\PY{n}{hydrobas\PYZus{}ww\PYZus{}p7}\PY{p}{,} \PY{n}{ppt\PYZus{}july\PYZus{}tif\PYZus{}pth}\PY{p}{,} \PY{n}{prefix}\PY{o}{=}\PY{l+s+s1}{\PYZsq{}}\PY{l+s+s1}{pptjuly\PYZus{}}\PY{l+s+s1}{\PYZsq{}}\PY{p}{,}
                                         \PY{n}{geojson\PYZus{}out}\PY{o}{=}\PY{k+kc}{True}\PY{p}{)}
\end{Verbatim}


    \begin{Verbatim}[commandchars=\\\{\}]
{\color{incolor}In [{\color{incolor} }]:} \PY{n+nb}{type}\PY{p}{(}\PY{n}{zonal\PYZus{}ppt\PYZus{}gjson}\PY{p}{)}\PY{p}{,} \PY{n+nb}{len}\PY{p}{(}\PY{n}{zonal\PYZus{}ppt\PYZus{}gjson}\PY{p}{)}
\end{Verbatim}


    \begin{Verbatim}[commandchars=\\\{\}]
{\color{incolor}In [{\color{incolor} }]:} \PY{n}{zonal\PYZus{}ppt\PYZus{}gdf} \PY{o}{=} \PY{n}{GeoDataFrame}\PY{o}{.}\PY{n}{from\PYZus{}features}\PY{p}{(}\PY{n}{zonal\PYZus{}ppt\PYZus{}gjson}\PY{p}{)}
        \PY{n}{zonal\PYZus{}ppt\PYZus{}gdf}\PY{o}{.}\PY{n}{head}\PY{p}{(}\PY{l+m+mi}{2}\PY{p}{)}
\end{Verbatim}


    \hypertarget{and-finally-a-choropleth-map-of-july-precipitation-by-watershed-with-a-good-bit-of-plot-tweaking.}{%
\paragraph{And finally, a choropleth map of July precipitation by
watershed! With a good bit of plot
tweaking.}\label{and-finally-a-choropleth-map-of-july-precipitation-by-watershed-with-a-good-bit-of-plot-tweaking.}}

    \begin{Verbatim}[commandchars=\\\{\}]
{\color{incolor}In [{\color{incolor} }]:} \PY{n}{f}\PY{p}{,} \PY{n}{ax} \PY{o}{=} \PY{n}{plt}\PY{o}{.}\PY{n}{subplots}\PY{p}{(}\PY{l+m+mi}{1}\PY{p}{,} \PY{n}{figsize}\PY{o}{=}\PY{p}{(}\PY{l+m+mi}{8}\PY{p}{,} \PY{l+m+mi}{6}\PY{p}{)}\PY{p}{)}
        \PY{n}{ax}\PY{o}{.}\PY{n}{set\PYZus{}title}\PY{p}{(}\PY{l+s+s1}{\PYZsq{}}\PY{l+s+s1}{Mean July precipitation (\PYZdl{}mm/month\PYZdl{}) by watershed}\PY{l+s+s1}{\PYZsq{}}\PY{p}{)}
        \PY{n}{zonal\PYZus{}ppt\PYZus{}gdf}\PY{o}{.}\PY{n}{plot}\PY{p}{(}\PY{n}{ax}\PY{o}{=}\PY{n}{ax}\PY{p}{,} \PY{n}{column}\PY{o}{=}\PY{l+s+s1}{\PYZsq{}}\PY{l+s+s1}{pptjuly\PYZus{}mean}\PY{l+s+s1}{\PYZsq{}}\PY{p}{,} \PY{n}{scheme}\PY{o}{=}\PY{l+s+s1}{\PYZsq{}}\PY{l+s+s1}{Equal\PYZus{}Interval}\PY{l+s+s1}{\PYZsq{}}\PY{p}{,} \PY{n}{k}\PY{o}{=}\PY{l+m+mi}{5}\PY{p}{,} 
                           \PY{n}{cmap}\PY{o}{=}\PY{n}{plt}\PY{o}{.}\PY{n}{cm}\PY{o}{.}\PY{n}{Blues}\PY{p}{,} \PY{n}{linewidth}\PY{o}{=}\PY{l+m+mi}{1}\PY{p}{,} \PY{n}{edgecolor}\PY{o}{=}\PY{l+s+s1}{\PYZsq{}}\PY{l+s+s1}{black}\PY{l+s+s1}{\PYZsq{}}\PY{p}{,} 
                           \PY{n}{legend}\PY{o}{=}\PY{k+kc}{True}\PY{p}{,} \PY{n}{legend\PYZus{}kwds}\PY{o}{=}\PY{p}{\PYZob{}}\PY{l+s+s1}{\PYZsq{}}\PY{l+s+s1}{loc}\PY{l+s+s1}{\PYZsq{}}\PY{p}{:} \PY{l+s+s1}{\PYZsq{}}\PY{l+s+s1}{upper left}\PY{l+s+s1}{\PYZsq{}}\PY{p}{\PYZcb{}}\PY{p}{)}
        \PY{n}{ax}\PY{o}{.}\PY{n}{set\PYZus{}facecolor}\PY{p}{(}\PY{l+s+s2}{\PYZdq{}}\PY{l+s+s2}{lightgray}\PY{l+s+s2}{\PYZdq{}}\PY{p}{)}
        \PY{n}{plt}\PY{o}{.}\PY{n}{axis}\PY{p}{(}\PY{l+s+s1}{\PYZsq{}}\PY{l+s+s1}{equal}\PY{l+s+s1}{\PYZsq{}}\PY{p}{)}\PY{p}{;}
\end{Verbatim}


    \hypertarget{other-resources-tools-and-overlap-with-other-tutorials}{%
\subsection{9. Other resources, tools, and overlap with other
tutorials}\label{other-resources-tools-and-overlap-with-other-tutorials}}

\begin{itemize}
\tightlist
\item
  Advanced vector geospatial analytics

  \begin{itemize}
  \tightlist
  \item
    Go over all the analytics that have been wrapped into GeoPandas,
    including
    \href{http://geopandas.org/geometric_manipulations.html}{Geometric
    Manipulations},
    \href{http://geopandas.org/set_operations.html}{Set-Operations with
    Overlay},
    \href{http://geopandas.org/aggregation_with_dissolve.html}{Aggregation
    with Dissolve} and
    \href{http://geopandas.org/aggregation_with_dissolve.html}{Merging
    Data}
  \item
    Use \href{http://pysal.org}{PySAL}, the Python Spatial Analysis
    Library! It's in the conda environment (version 1.14.4). This is a
    powerful, multi-faceted package. But watch out for the current
    transition from ``version 1'' to ``version 2''.
  \end{itemize}
\item
  Visualizations

  \begin{itemize}
  \tightlist
  \item
    \href{https://residentmario.github.io/geoplot/index.html}{GeoPlot}
    makes it easy to generate a variety of useful plots from
    GeoDataFrames. It's available in the conda environment.
    \href{http://geopandas.org/gallery/plotting_with_geoplot.html}{Here
    is a gallery of GeoPlot plots from GeoPandas}
  \item
    More spatial visualization options are coming in the
    \href{https://geohackweek.github.io/visualization/}{Visualization
    tutorial}. Stay tuned, or look it up now. The mapping packages it'll
    cover include
    \href{https://scitools.org.uk/cartopy/docs/latest/}{cartopy},
    \href{http://geoviews.org/}{GeoViews} and
    \href{http://python-visualization.github.io/folium/}{Folium} (we
    only covered a small subset of Folium capabilities, just to give you
    a taste)
  \item
    See the
    \href{https://github.com/oceanhackweek/ohw2018_tutorials/tree/master/day3/geospatial_and_mapping_tools}{Python
    mapping libraries tour} from OceanHackWeek 2018 last month.
  \end{itemize}
\item
  Overlap with raster processing

  \begin{itemize}
  \tightlist
  \item
    We illustrated \texttt{rasterstats} and
    \href{https://rasterio.readthedocs.io/en/latest/}{rasterio}.
    \texttt{rasterio} will be a pretty important component of your
    raster handling and manipulation toolbox. And it interacts with the
    GeoJSON-like objects we've examined; for example, see its
    \href{https://rasterio.readthedocs.io/en/latest/topics/features.html}{features
    module}.
  \item
    \href{https://regionmask.readthedocs.io/}{regionmask} works nicely
    with GeoDataFrames to support gridded operations, including ones in
    \texttt{xarray} that you'll see in the
    \href{https://geohackweek.github.io/nDarrays/}{nDarrays tutorial}
  \end{itemize}
\end{itemize}


    % Add a bibliography block to the postdoc
    
    
    
    \end{document}
